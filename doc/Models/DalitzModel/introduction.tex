% !TEX root = notes.tex
\section{Introduction}
\label{sec:introduction}

When an experiment publishes a new amplitude analysis of a decay it often follows that it has used it's own parameterisation for the models involved in the amplitude analysis.
This can be troublesome when trying to recreate amplitude models for various studies. In this we aim to document some of the parameterisations used by different experiments
so they can be reproduced.

\subsection{Phenomenological parameterization of the D0 decay amplitude}

A three-body decay $\decay{X}{ABC}$ can be described as a superposition of quasi-two-body decays $\decay{X}{RC}$, $\decay{R}{AB}$. 
We can first describe the general expressiom for the decay of a particle $X$ into daughters $ABC$ amplitude

\begin{equation}
    d\Gamma = \frac{\bar{|\mathcal{M}|^{2}}}{32(2\pi)^{3}M_{X}^{3}}dm_{BC}^{2}dm_{AC}^{2},
\end{equation}

where $\mathcal{M}$ is the amplitude at some particular point in phase space, and $\bar{|\mathcal{M}|^{2}}$ is the amplitude averaged over the spin states of $X$.

Let us consider $X$ being a $\Dz$ meson, and look at its amplitude. We can write the amplitude as

\begin{align}
    \mathcal{M}\left( ABC | R \right) & = \sum_{m_{\lambda}}\langle AB | R_{m_{\lambda}} \rangle  T_{R}\left( \sqrt{ s_{AB} } \right) \langle CR | \Dz \rangle \\
    & = Z_{\lambda}\left( \vec{p}, \vec{q} \right) B_{\Dz}^{'L} \left( p \right) B_{R}^{'L} \left( q \right) T_{R} \left( \sqrt{s_{AB}} \right).
\end{align}

Here $\vec{q}$ and $\vec{p}$ are the momenta of $A$ and $C$ in the $R$ and $X$ rest frame respectively. Each term will be described in more detail presently.

When more than one resonance $R$, comtributes to the decay, we sum coherently over the amplitudes for all the intermediate resonances,

\begin{equation}
    \mathcal{M}\left( ABC \right) = c_{0} + \sum_{R} c_{R} \mathcal{M} \left( ABC | R \right),
\end{equation}

where $c_{j}$ are the corresponding complex coefficients, and $c_{0}$ is a complex constant modeling no resonant contributions, which is often zero.

The function $Z_{\lambda}\left( \vec{p}, \vec{q} \right)$ describes the angular distribution of the decay products, whule $B_{i}^{L}$ are the form factors, which are commenly parameterised using the Blatt-Weisskopf penetration for factors.

Finally $T_{R}\left( \sqrt{s_{AB}} \right)$ is the propagat, this is where most experiment differ in there parameterisations.