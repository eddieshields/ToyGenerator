%\pdfoutput=1
% Uncomment line above if submitting to arXiv and using pdflatex

\documentclass[12pt,a4paper]{article}
% For two column text, add "twocolumn" as an option to the document
% class. Also uncomment the two "onecolumn" and "twocolumn" lines
% around the title page below.

% Variables that controls behaviour
\usepackage{ifthen} % for conditional statements
\newboolean{pdflatex}
\setboolean{pdflatex}{true} % False for eps figures

\newboolean{articletitles}
\setboolean{articletitles}{true} % False removes titles in references

\newboolean{uprightparticles}
\setboolean{uprightparticles}{false} %True for upright particle symbols

\newboolean{inbibliography}
\setboolean{inbibliography}{false} %True once you enter the bibliography

% Define titles and authors here. It will then be used both in metadata and in
% what is printed on the front page.
\def\paperauthors{Edward Shields} % Leave as is for PAPER and CONF
\def\paperasciititle{A survey of parameterisations used in amplitude analysis} % Set ASCII title here
\def\papertitle{A survey of parameterisations used in amplitude analysis} % Latex formatted title
\def\paperkeywords{{High Energy Physics}, {LHCb}} % Comma separated list
%\def\papercopyright{CERN on behalf of the LHCb collaboration}
\def\papercopyright{\the\year\ CERN}% for the benefit of the LHCb collaboration} % new since 9/Apr/2018
\def\paperlicence{CC-BY-4.0 licence}
\def\paperlicenceurl{https://creativecommons.org/licenses/by/4.0/}

\input{preamble}
\usepackage{longtable} % only for template; not usually to be used in PAPERs
\usepackage{booktabs}
\usepackage{multirow}
\usepackage{braket} % for formalism
\usepackage[makeroom]{cancel}
\usepackage{mathtools}
% \usepackage{catchfile}
% \newcommand{\getenv}[2][]{%
%   \CatchFileEdef{\temp}{"|kpsewhich --var-value #2"}{}%
%   \if\relax\detokenize{#1}\relax\temp\else\let#1\temp\fi}

\begin{document}

%%%%%%%%%%%%%%%%%%%%%%%%%
%%%%% Title     %%%%%%%%%
%%%%%%%%%%%%%%%%%%%%%%%%%
\renewcommand{\thefootnote}{\fnsymbol{footnote}}
\setcounter{footnote}{1}

% %%%%%%% CHOOSE TITLE PAGE--------
%\onecolumn
\input{titlepage}
%\twocolumn
% %%%%%%%%%%%%% ---------

\renewcommand{\thefootnote}{\arabic{footnote}}
\setcounter{footnote}{0}

%%%%%%%%%%%%%%%%%%%%%%%%%%%%%%%%
%%%%%  Table of Content   %%%%%%
%%%%%%%%%%%%%%%%%%%%%%%%%%%%%%%%
%%%% Uncomment next 2 lines if desired
\tableofcontents
\cleardoublepage


%%%%%%%%%%%%%%%%%%%%%%%%%
%%%%% Main text %%%%%%%%%
%%%%%%%%%%%%%%%%%%%%%%%%%

\pagestyle{plain} % restore page numbers for the main text
\setcounter{page}{1}
\pagenumbering{arabic}

%% Uncomment during review phase.
%% Comment before a final submission.
\linenumbers

% You can include short sections directly in the main tex file.
% However, for larger papers it is desirable to split the text into
% several semiautonomous files, which can be revised independently.
% This is especially useful when developing a document in
% collaboration with several people, since then different parts can be
% edited independently.  This type of file organization is shown here.
%

% Versions info
%\input{history}
%\newpage

% Main text
% !TEX root = notes.tex
\section{Introduction}
\label{sec:introduction}

When an experiment publishes a new amplitude analysis of a decay it often follows that it has used it's own parameterisation for the models involved in the amplitude analysis.
This can be troublesome when trying to recreate amplitude models for various studies. In this we aim to document some of the parameterisations used by different experiments
so they can be reproduced.

\subsection{Phenomenological parameterization of the D0 decay amplitude}

A three-body decay $\decay{X}{ABC}$ can be described as a superposition of quasi-two-body decays $\decay{X}{RC}$, $\decay{R}{AB}$. 
We can first describe the general expressiom for the decay of a particle $X$ into daughters $ABC$ amplitude

\begin{equation}
    d\Gamma = \frac{\bar{|\mathcal{M}|^{2}}}{32(2\pi)^{3}M_{X}^{3}}dm_{BC}^{2}dm_{AC}^{2},
\end{equation}

where $\mathcal{M}$ is the amplitude at some particular point in phase space, and $\bar{|\mathcal{M}|^{2}}$ is the amplitude averaged over the spin states of $X$.

Let us consider $X$ being a $\Dz$ meson, and look at its amplitude. We can write the amplitude as

\begin{align}
    \mathcal{M}\left( ABC | R \right) & = \sum_{m_{\lambda}}\langle AB | R_{m_{\lambda}} \rangle  T_{R}\left( \sqrt{ s_{AB} } \right) \langle CR | \Dz \rangle \\
    & = Z_{\lambda}\left( \vec{p}, \vec{q} \right) B_{\Dz}^{'L} \left( p \right) B_{R}^{'L} \left( q \right) T_{R} \left( \sqrt{s_{AB}} \right).
\end{align}

Here $\vec{q}$ and $\vec{p}$ are the momenta of $A$ and $C$ in the $R$ and $X$ rest frame respectively. Each term will be described in more detail presently.

When more than one resonance $R$, comtributes to the decay, we sum coherently over the amplitudes for all the intermediate resonances,

\begin{equation}
    \mathcal{M}\left( ABC \right) = c_{0} + \sum_{R} c_{R} \mathcal{M} \left( ABC | R \right),
\end{equation}

where $c_{j}$ are the corresponding complex coefficients, and $c_{0}$ is a complex constant modeling no resonant contributions, which is often zero.

The function $Z_{\lambda}\left( \vec{p}, \vec{q} \right)$ describes the angular distribution of the decay products, whule $B_{i}^{L}$ are the form factors, which are commenly parameterised using the Blatt-Weisskopf penetration for factors.

Finally $T_{R}\left( \sqrt{s_{AB}} \right)$ is the propagat, this is where most experiment differ in there parameterisations.

\section{Angular distribution}
\label{sec:angular}

There are two formalisms to be considered for the angular distribution: Zemach and Helicity.
The Zemach formalism is as follows,

\begin{align}
    Z_{0} & = 1, \\
    Z_{1} & = m_{AC}^{2} - m_{BC}^{2} + \frac{\left( m_{\Dz}^{2} - m_{C}^{2} \right) \left( m_{B}^{2} - m_{A}^{2} \right)}{m^{2}_{AB}}, \\
    Z_{2} & = -\frac{1}{3} \left(  m_{AB}^{2} - 2m_{\Dz}^{2} - 2m_{C}^{2} + \frac{\left( m_{\Dz}^{2} - m_{C}^{2} \right)^{2}}{m_{AB}^{2}}  \right) 
    \left(  m_{AB}^{2} - 2m_{A}^{2} - 2m_{B}^{2} + \frac{\left( m_{A}^{2} - m_{B}^{2} \right)^{2}}{m_{AB}^{2}}  \right) + 
    \left( m_{BC}^{2} - m_{AC}^{2} + \frac{\left( m_{\Dz}^{2} - m_{C}^{2} \right) \left( m_{A}^{2} - m_{B}^{2} \right)}{m^{2}_{AB}} \right)^{2}
\end{align}

as is found in~\cite{PhysRevLett:89:251802}. For the Helicity formalism $m_{R}^{2}$ is used in place of $m_{AB}^{2}$.
\subsection{Blatt-Weisskopf penetration factors}

The common parameterisation of the Blatt-Weisskopf form factors is,

\begin{align}
    B_{0}\left( q \right) & = 1, \\
    B_{1}\left( q \right) & = \sqrt{ \frac{2z}{z+1} }, \\
    %B_{1}\left( q \right) & = \frac{13z^{2}}{\left( z - 3 \right)^{2} + 9z},
\end{align}
%
%where $z = q\left( s_{AB} \right)R$ and $R$ is most commenly taken to be $1.5\text{GeV}^{-1}$.
%
%Additionally we can define the relative form factors,
%
%\begin{align}
%    B_{0}^{'}\left( q, q_{0} \right) & = 1, \\
%    B_{1}^{'}\left( q, q_{0} \right) & = \frac{2z}{z+1}, \\
%    B_{1}^{'}\left( q, q_{0} \right) & = \frac{13z^{2}}{\left( z - 3 \right)^{2} + 9z},
%\end{align}

\addcontentsline{toc}{section}{References}
\setboolean{inbibliography}{true}
\bibliographystyle{LHCb}
\bibliography{notes}


%\newpage



\end{document}
